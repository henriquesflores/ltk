\documentclass[12pt]{article}

%\usepackage[a4paper, margin=2cm]{geometry}
\usepackage[utf8]{inputenc}
\usepackage[T1]{fontenc}
\usepackage{hyperref}
\usepackage{amsmath}
\usepackage{amsfonts}
\usepackage{amssymb}
\usepackage{amsthm}
\usepackage{amsbsy}
\usepackage{mathrsfs}
\usepackage{slashed}
\usepackage{graphicx}
\usepackage{fancyhdr}
\usepackage{color}
\usepackage{cite}
\usepackage{float}

\numberwithin{equation}{section}

\newtheoremstyle{henrique}      % Name
  { }                           % Space above
  {}                            % Space below
  {\normalfont}                 % Body font
  {}                            % Indent amount
  {\bfseries}                   % Theorem head font
  {.}                           % Punctuation after theorem head
  { }                           % Space after theorem head, ' ', or \newline
  {}                            % Theorem head spec (can be left empty, meaning `normal')
\theoremstyle{henrique}
\newtheorem{teo}{Theorem}[section]
\newtheorem{defi}{Definition}[section]
\newtheorem{lema}{Lemma}[section]
\newtheorem{corol}{Corollary}[section]
\newtheorem{prop}{Proposition}[section]


\hypersetup{colorlinks=false, 
            urlbordercolor={1 0 0}, 
            linkbordercolor={0 0 1}, 
            citebordercolor={0 0 1} }

\usepackage[a4paper, margin=2cm]{geometry}
\usepackage[utf8]{inputenc}
\usepackage[T1]{fontenc}
\usepackage{hyperref}
\usepackage{amsmath}
\usepackage{amsfonts}
\usepackage{amssymb}
\usepackage{amsthm}
\usepackage{amsbsy}
\usepackage{mathrsfs}
\usepackage{slashed}
\usepackage{graphicx}
\usepackage{fancyhdr}
\usepackage{color}
\usepackage{cite}
\usepackage{float}

\numberwithin{equation}{section}

\newtheoremstyle{henrique}      % Name
  { }                           % Space above
  {}                            % Space below
  {\normalfont}                 % Body font
  {}                            % Indent amount
  {\bfseries}                   % Theorem head font
  {.}                           % Punctuation after theorem head
  { }                           % Space after theorem head, ' ', or \newline
  {}                            % Theorem head spec (can be left empty, meaning `normal')
\theoremstyle{henrique}
\newtheorem{teo}{Theorem}[section]
\newtheorem{defi}{Definition}[section]
\newtheorem{lema}{Lemma}[section]
\newtheorem{corol}{Corollary}[section]
\newtheorem{prop}{Proposition}[section]


\hypersetup{colorlinks=false, 
            urlbordercolor={1 0 0}, 
            linkbordercolor={0 0 1}, 
            citebordercolor={0 0 1} }


\usepackage{amssymb}

\title{Notes on Probability Theory}
\author{Henrique Flores \\ M.Safra \& Co.}
\date{}

% Overcounting

\begin{document}
\maketitle
\tableofcontents

\section{Definitions}
There is a set $\Omega$ called sample space.
Every subset inside $\Omega$ is called an event and probabilities are
functions $p$ of events to the real interval $[0, 1]$, that is, $p: A \subset \Omega \to [0,1]$.
The axioms of probability theory can be summarized as follows:

\begin{itemize}
\item $P(\Omega) = 1$ and $P(\emptyset) = 0$, where $\emptyset$ is the empty set.
\item $P \left( \cup_{n = 1}^{\infty} A_{n} \right) = \sum_{n = 1}^{\infty} P(A_{n})$ if
and only if $A_{i}$ are mutually disjoint for $i = 1, \ldots, n$.
\end{itemize}

\noindent
Every probability rule can be derived from these two axioms and set operations.
Let us prove $P(A) = 1 - P(A^{c})$ where $A^{c}$ is the complement of $A$.
It follows immediatelly from the considerations that $ A \cup A^{c} = \Omega$ in such a way that
$A$ and $A^{c}$ are disjoint, and that $P(\Omega) = 1$.
We do not intend to prove every formula written in this text; 
only those which do not appear intuitive or represent a good exercise.

For $A$ and $B$ inside $\Omega$ not necessarily disjoint, it is true that 

\begin{equation}
p(A \cup B) = p(A) + p(B) - p(A \cap B),
\end{equation}

\noindent
which states that probabilities are linear; the last term
discounts the contribution of one intersection so there is no overcounting. 
This is the particular case of the more general \textit{inclusion and exclusion}
principle.

For intersections, we define the quantity $P(A | B)$ by

\begin{equation} \label{intersection1}
p(A \cap B) = p( A | B ) p( B ),
\end{equation}

\noindent
to be read as: \textit{the probability of $A$ given that $B$},
or the probability of event $A$ given that we event $B$ has already happened.
On intersections, we are conditioning the probability of one event to the other,
because that is precisely what it means to intersect. The probability of $A$ given $B$
is modulated by the probability of $B$ happening.
Since there is no particular reason that $A$ and $B$ on their intersection should be distinguishable,
we can also write

\begin{equation} \label{intersection2}
p(A \cap B) = p( B | A ) p (A).
\end{equation}

\noindent
Consistency of (\ref{intersection1}) and (\ref{intersection2}) gives us 
Bayes rule:

\begin{equation}
p( A | B ) p (B) = 
p( B | A ) p (A)
\end{equation}

Of extreme importance for probability theory is the law of marginalization,

\begin{equation}
P(A) = \sum_{B} P(A | B) P(B),
\end{equation}

\noindent
where $B$ form a disjoint partition of $\Omega$.
Let us prove this relation.
Consider a disjoint partition of $\Omega$, that is, a collection of disjoint
subsets $B_{i}$ such that $ \bigcup_{i = 1}^{N} B_{i} = \Omega$.
Whether or not is possible to find such partitions is a consideration of 
\textit{topology}, which is beyond the scope of these notes.
We will work under the assumption that such partitions always exist. 
Given such partition, it follows:

\begin{equation}
A \cap \left( \bigcup_{i = 1}^{N} B_{i} \right) = \bigcup_{i = 1}^{N} \left( A \cap B_{i} \right).
\end{equation}

\noindent
Therefore

\begin{align}
P(A) &= P(A \cap \Omega)
\nonumber \\
\nonumber \\
&= P \left( A \cap \bigcup_{i = 1}^{N} B_{i} \right) 
= P \left( \bigcup_{i=1}^{N}  A \cap B_{i} \right) 
\nonumber \\
\nonumber \\
&= \sum_{i = 1}^{N} P \left( A \cap B_{i} \right) 
\nonumber \\
\nonumber \\
&= \sum_{i = 1}^{N} P \left( A | B_{i} \right) P(B_{i})
\end{align}

\noindent
where we have used the fact that the intersections $A \cap B_{i}$ are disjoint
for each $i$ given that $B_{i}$ are disjoint.
\paragraph{Monty Hall problem.}
Let us understand Bayes rule solving the Monty Hall problem.
There are $3$ doors: two of them contain garbage but one contains a special prize.
You have to choose one.
Given your choice, a door that contains garbage and it is not the one you have choosen is revealed to you.
There are only two doors now and you are given the chance to change the door you picked. Do you switch?

\paragraph{Answer.}
Let $H$ represent the prize location: $H = 1$ means that the prize is inside door $1$, \textit{etc}.

At the beginning every door is equally likely, so the prior probability is

\begin{equation}
p(H = 1) = p (H = 2) = p (H = 3 ) = 1/3.
\end{equation}

Let us choose door $3$, and it is revealed that door $1$ is garbage.
Now we have to compute $p ( H = 3 | D = 1 )$, that is,
what is the probability that the prize is on door $3$ given that we know 
door $1$ is gargage.
We will solve this applying Bayes rule:

\begin{equation}
p ( H = 3 | D = 1 ) = \frac{ p ( D = 1 | H = 3 ) p ( H = 3 ) }{ p ( D = 1 ) }.
\end{equation}

In order to compute the probability that the door opened is $1$ given that the prize is on door $3$ we
use the following table:

\begin{center}
\begin{tabular}{| c | c | c | c |}
\hline
& D = 1 & D = 2 & D = 3 \\
\hline
H = 1 & 0 & 1 & 0 \\
\hline 
H = 2 & 1 & 0 & 0 \\
\hline
H = 3 & 1/2 & 1/2 & 0 \\
\hline
\end{tabular}
\end{center}

\noindent
Notice that the column $D = 3$ is entirely zero because this door can never be opened since it contains the prize.
The probabilities are then

\begin{equation}
p ( H = 3 | D = 1 ) =  1/3 * 1/2 * 2 = 1/3
\end{equation}

\noindent
together with

\begin{equation}
p ( H = 1 | D = 1 ) = 0
\end{equation}

\noindent
and

\begin{equation}
p ( H = 2 | D = 1 ) = 2/3
\end{equation}

\noindent
So, the probability that the prize is really behind door number $3$ given that the 
door $1$ is open is really just $1/3$.
We are better off switching doors.

I want to make two small comments.
The first comment is that the problem is symmetric. 
If we had choosen instead of door $3$ door number $2$,
the probabilities would have not changed.
What changes is the conditional table:

\begin{center}
\begin{tabular}{| c | c | c | c |}
\hline
& D = 1 & D = 2 & D = 3 \\
\hline
H = 1 & 0 & 0 & 1 \\
\hline 
H = 2 & 1/2 & 0 & 1/2 \\
\hline
H = 3 & 1 & 0 & 0 \\
\hline
\end{tabular}
\end{center}

\noindent
Combining everything, we arrive at

\begin{equation}
p( H = 3 | D = 1 ) = 1 * 1/3 * 2 = 2/3
\end{equation}

\noindent
together with

\begin{equation}
p ( H = 1 | D = 1 ) = 0
\end{equation}

\noindent
and 

\begin{equation}
p ( H = 2 | D = 1 ) = 1/3.
\end{equation}

\noindent
And again we are better off switching.

The last comment is about the probability that door $1$ is opened in either scenario. 
I do not have an intuitive explanation for the value of $1/2$.
But this value has been computed with the law of marginalization:

\begin{equation}
p(D = 1) = \sum_{H = 1}^{3}  p(D = 1 | H ) p(H)
\end{equation}

\vspace{.5cm}
Another important example of Bayes theorem is the 
probability of having a rare disease.

\paragraph{Probability of having a disease.}
You suspect that you have a very rare disease which occurs with a frequency of $0.001$ in the population.
After performing a test with $99\%$ of confidence that results positive, what is the probability that 
you actually have the disease?

\paragraph{Answer.}
We wish to compute $p( H = \mathrm{True} | E = \mathrm{True} )$, which is read as
the probability that the hypothesis $H$ of having the disease is true when the event $E$ is also positive.

Via Bayes theorem 

\begin{equation}
p ( H = \mathrm{True} | E = \mathrm{True} ) = \frac{ p( E = \mathrm{True} | H = \mathrm{True} ) p ( H = \mathrm{True} ) } { p ( E = \mathrm{True} ) }
\end{equation}

\noindent
the probability that we are interested in depends on the probability that the test returns positive given that it is true
that we have the disease, which is the degree of accuracy of the test: $99\%$, and on the probability of having the disease 
in the first place, for which the frequency in population is the best estimate.
For the probability of the test returning positive, we use the law of marginalization:

\begin{equation}
p( E = \mathrm{True} ) = \sum p ( E = \mathrm{True} | H ) p (H).
\end{equation}

Plugging in the numbers, we obtain:

\begin{equation}
p ( H = \mathrm{True} | E = \mathrm{True} ) = 
\frac{ 0.99 \times 0.001 }{ 0.99 \times 0.001 + 0.01 \times 0.999 } = 9 \% .
\end{equation}

\noindent
At first glance, $9\%$ seems so small, how could this be?
The point is that the frequency in the population is so small that for every $1000$ people
that are tested we have 10 false positives for 1 occurence of the disease.
This means that false positives are more frequent than being sick, and so our probability of 
having the disease after testing positive is actually small.

If we run the test a second time 

\begin{equation}
p ( H = \mathrm{True} | E = \mathrm{True} ) = 
\frac{ 0.99 \times 0.09 }{ 0.99 \times 0.09 + 0.01 \times 0.91 } = 91 \% 
\end{equation}

\noindent
where we have now updated our belief that we actually have the disease for the calculated $9 \%$.

This number now seems more likely. But interestingly enough, they are still not as high as the 
test confidence.

\section{Counting.}

Several problems in probability can be solved by counting.
Counting methods apply to experiments with a finite number of outcomes.
This section revises the basic intuition behind the counting formulas.

Let us build our intuition going to increasingly complicated examples.
Consider a list of $26$ different elements: $\{ 1, \ldots, 26 \}$.
Let us begin calculating how many sequences of 3 elements can we form if we allow repetitions?
Well, for each place in the sequence, we can choose any number between $1$ and $26$, giving a total
of $26$ possibilities for each slot. Therefore we have $26 \times 26 \times 26$ different possibilites.
What if we do not allow repetitions to take place? Then once a given element is picked, it cannot be picked
again, so out of $26$ possibilities in the first slot, we will have $25$ for the second since one element was 
already choosen. In total, we obtain $26 \times 25 \times 24$ combinations that do not have repetitions.
Notice that the difference of $26^{3}$ and $26 \times 25 \times 24$ gives $1976$, which is the number of sequences
that have at least one repeated element.

The number of sequences that have at least one repeated element can be counted in a different way.
Consider that once we have filled the first spot, the second spot is known leaving the third one free. We have then $26 \times 25$ possibilities.
But we could have filled the first spot and the third spot leaving the second spot free, giving again $26 \times 25$ different possibilites.
Finally, we could fill the second and third spots with the same element, leaving the first one free to vary over the $26$ possibilites.
In total, if we have one repetition, there are $3 \times 26 \times 25$ possibilites.
If we have two repetitions, then we only have $26$ possibilities since fixing one spot, fixes all remaining two.
Therefore, the number of sequences that have at least one repeated element is given by $3 \times 26 \times 25 + 26 = 1976$.

\subsection{Permutations.}

Let us now generalize a bit by considering a list of $n$ elements: $\{ 1, \ldots, n \}$.
We wish to answer the following question: How many sequences without repetition can we form?
Let us name a given sequence with no repeated elements a permutation of the sequence $\{1, \ldots, n \}$.
For example, two possible permutations are: $(1, 3, 2, \ldots, n )$ and $(1, 2, 3, \ldots, n, n - 1)$.
Then, considering the counting without repetition, the number of permutations is 
$n \times (n - 1) \times (n - 2) \times \ldots \times 1$ and we denote this by $n!$:

\begin{equation}
n! = n \times (n - 1) \times (n - 2) \times \ldots \times 1
\end{equation}

\subsection{Number of subsets.}

Given a set with $n$ elements, what is the number of all possible subsets?
That is, given the sequence $\{1, \ldots, n\}$ we have to choose if we select or not
$1$, if we select or not $2$, \textit{etc}.
Therefore we have for each element, a binary choice of whether we put it or not the element inside the subset.
This gives

\begin{equation}
\# \mbox{ of subsets } =
2^{n}
\end{equation}

\paragraph{Example.}
We roll six, six-sided, dies; what is the probability that all six give different numbers?
It is intuitive that each die is independent of each other, so we can consider that we have 
$6^{6}$ possible configurations. 
The outcomes in which all dies land on different sides is given by 
$6!$, and finally our answer is $ 6! / 6^{6}$.

\vspace{.5cm}
Up to now probability is computed using the number of configurations that we are interested in divided
by the total number of possible configurations that the problem can have. 
This procedure assumes that all possible events are equally likely.
Moreover, if we wish equally likely events, then the frequency of each possible configuration is 
counted as if every possible element in the problem is distinguishable.
For example, if we again consider the $6$ dices and ask how many configurations have at least one dice with $1$
facing up, we pretend that we can distinguish among the $6$ dices for couting purposes, even though
when we look at the dices on the table, it would be impossible to tell.
We fix one dice to be $1$ and the other $5$ can have any of the $5$ possible configurations.
Therefore, we have the location of the $1$-dice and the configuration of each of the remaining
ones, giving in total, $6 \times 5^{5}$ possibilities.

\subsection{Combination.}
Out of our set of $n$ elements, we wish to pick $k$ of them. 
For example, in the set $\{1, 2, 3\}$, if we wish to choose $2$ elements. All possible choices
are given by $\{ (1 , 2), (1 , 3), (2, 3) \}$. Notice that we consider equivalent permutations of the pair, 
that is, $(1, 2) \approx (2, 1)$.
The solution to this problem can be obtained as follows. Since $k \leq n$ than let us write all possible
permutations of $n$ elements and then pick the first $k$ elements that appear.
We \textbf{are not interested} in the order in which we pick the $k$ elements, so we have to discount
the permutations of $k$. This gives $n! / k!$, but the same elements can be picked by us if the permutation
only shuffles the remaning $(n - k)$ elements. Therefore this number must also be discounted.
Finally, let us denote ${n \choose k}$ -- to be read \textit{$n$ choose $k$} -- the number of
combinations in which we can choose $k$ elements out of $n$:

\begin{equation}
{n \choose k} = \frac{n!}{k! (n - k)!}.
\end{equation}

This result can be easily generalized for the case where we consider the different ways we pick $k$ elements:
just remove the $k!$ from the denominator.

\paragraph{Example.}
Consider independent coin tosses where $p(H) = p$. What is the probability that out of $n$ tosses
we obtain $k$ heads $H$.

Well, each particular configuration of heads and tails has the probability $P$ given by

\begin{equation}
P = p^{\# \, \mbox{ of heads}} ( 1 - p)^{\# \, \mbox{of tails}} = p^{k} (1 - p)^{n - k}.
\end{equation}

\noindent
In the last line we have substituted the number of heads and tails given in the problem.
The final point is that out of $10$ tosses, we could have obtained $5$ heads in the first five
tosses or in the last five tosses. Each of these configurations has probability given by $P$, but
there are ${n \choose k}$ different ways that this specific configuration of $k$ heads in $n$ tosses
can happen.
Thus, the probability of obtaining $k$ heads out of $n$ tosses is given by

\begin{equation}
p( k | n ) = {n \choose k} p^{k} ( 1 - p )^{n - k}
\end{equation}

It is interesting to ask what happens if we add the probability of having one heads with the probability of having two heads and so on.
Intuitively, we expect to get $1$ since we are adding over the probabilities of all combinations of heads appearing:

\begin{equation}
\sum_{k = 0}^{n}  {n \choose k} p^{k} ( 1 - p )^{n - k} = 1.
\end{equation}

\noindent
This result is a consequence of the binomial expansion.

\paragraph{Example II.}
You toss a coin $10$ times and someone tells you that there were $3$ heads.
What is the probability that the first two tosses gave heads?
%The probability that we have $3$ heads out of $10$ tosses is given by
%$p(3 | 10)$. If the first two tosses are heads, there remains only one heads
%to appear in any of the remaining $8$ positions. The probability that we have
%one heads in $8$ tosses is $p(1 | 8)$. Therefore our answer is 
%
%\begin{equation}
%p(1 | 8)/ p(3 | 10)
%\end{equation}
First, notice that all sequences of $3$ heads in $10$ tosses have the same probability of $p^{3} ( 1 - p)^{7}$.
We are interested in the number of ways that $3$ heads can appear in $10$ tosses ( ${10 \choose 3}$) and inside 
that set, what is the number of ways that one heads can be appear in $8$ tosses ( ${8 \choose 1}$ ) since 
the first two tosses are fixed to be heads.
Therefore, the probability is given by ${8 \choose 1}  / {10 \choose 3 }$.

\subsection{Partition.}
Given a set of $n$ elements, we wish to partition it into $k$ subsets of size $n_{1},  n_{2}, \ldots, n_{k}$
respectively. Again the same strategy used in the derivation of the combination formula can be used here.
We write a permutation of the set with $n$ elements and start partitioning it discouting the permutations 
inside the subset.
The answer is 

\begin{equation}
\frac{n!}{n_{1} ! n_{2} ! \ldots n_{k}!}
\end{equation}

\paragraph{Sampling with replacement.}
Let us sample $k$ times from a set of $n$ items. However in this
case we allow replacement. How many different picks there is?
For concreteness consider the specific problem of picking $n = 2$ balls $k$ times. 
We can go from never picking ball $1$ to picking it every single time,
giving a total of $\{0, \ldots, k\}$ different possible ways of sampling ball $1$, 
and a total of  $k + 1$ different ways.

The strategy for general $n$ is to consider that we have $n$ \textit{distinguishable}
boxes and each time we sample any ball, we are putting it in a given box.
For example, if the first sample picks ball $4$, we interpret this as
having placed a ball in box $n = 4$.
For $n$ boxes, we have $n-1$ boundaries, since each $2$ boundaries
defines a box. With this in mind, a possible configuration would be

\begin{equation}
\bullet \bullet \| \bullet | \bullet \bullet \bullet
\end{equation}

\noindent
where we have $2$ balls in the first box; an empty second box; a third box with
one ball; while the fourth, and final box, contains three balls.
One can see that we have $n + k -1$ spots where each spot is either a boundary or a 
ball. Given that the $k$ balls are indistinguishable, we finally obtain

\begin{equation}
\# \mbox{ ways of sampling with replacement} =
{ n + k - 1 \choose k }
\end{equation}

\noindent
We can test this general result for $n=2$:

\begin{equation}
{ k + 1 \choose k } = k + 1.
\end{equation}

\section{Discrete Random Variables.}

Random variables associate numerical variables to the outcomes of some experiment.	
At large, the introduction of random variables will just be a rewriting of the previous 
expressions.
Formally, random variables are functions from the sample space to the real numbers.
We follow the literature and denote a random variables capital letters $X$ while the 
numerical value they take are denoted by small letters $x$.

Given a random variable, some outcomes happen more often than others.
Usually one introduces the concept of probability mass function, which for every possible 
value $x$ gives the \textit{"frequency"} at which $x$ occurs, denoted $p_{X}(x)$ or $p(X = x)$.
In terms of set formulation of probability theory, the expression $p_{X}(x)$ can be thought as
the set $S \in \Omega$ for which $X(S) = x$.
It is very customary though to just write $X = x$ in such a way that the subset $S$ of $\Omega$ is 
implicit in the discussion.

\paragraph{Example.}
In a room there are $26$ people. We pick a person at random and measure its height $h$ and its weight
$w$. We have created two random variables $H$ and $W$ from the sample space of people with definite weights and heights
to the positive real numbers, the respective height and weight of each person.

\vspace{.5cm}
It is very natural to conclude that functions of random variables are simply functions applied to all of its possible outcomes,
that is, functions of functions.

\subsection{Probability mass function and cumulative distribution function.}

For discrete random variables, we define the probability mass function to be the 
probability that a random variable $X$ takes the value $x$, $p(X = x)$.
It must be between $0$ and $1$ and must also satisfy:

\begin{equation}
\sum_{x} p(X = x) = 1.
\end{equation}

\noindent
Conversely, if one can find a function that is always positive and has finite sum over all its possible 
values, one has defined a probability mass function. This trick is particularly interesting for functions
that have finite Taylor series.

For the remaining of this section we will discuss the most used probability mass functions.
These distributions are famous because they have a particularly nice story about themselves
which justifies their usefulness.

When we deal with continuous random variables though, the concept of probability mass function will 
not be very useful since it is impossible for a random variable to have any particularly fixed
real value with infinite arbitrary precision, which means $p(X = x) = 0$ for all $x \in S$.
A concept that is generalizable is called cumulative distribution function, which we denote $F(x)$, 
and is defined by

\begin{equation}
F(x) = P( X \leq x).
\end{equation}

\noindent
Both probability mass functions (PMF) and the cumulative distribution function (CDF)
contain the exact same information about the distribution in question.

Later we will consider in more detail the probability mass functions of several random variables,
which are functions taking values over $[0, 1]$  that depend on two or more random variables.
For the moment, let us consider two random variables $X$ and $Y$.
Their \textit{joint} probability distribution is denoted $P(X = x \text{ and } Y = y) = P(X = x, Y = y)$;
it represents the probability of each random variable acquiring a particular 
value, and it is related to the probabilities of indiviadual the random variables via marginalization:

\begin{subequations}
\begin{align}
&\sum_{y} P(X = x, Y = y) = P(X = x)
\\
\nonumber \\
&\sum_{x} P(X = x, Y = y) = P(Y = y)
\end{align}
\end{subequations}

When dealing with multiple random variables, it is 
necessary to understand whether they are dependent or 
independent.
We define independence of two events $A$ and $B$ by

\begin{equation}
P(A | B) = P(A),
\end{equation}

\noindent
that is knowing that the event $B$ happened does not give you any new information
on the event $A$. You do not have to revise your previous beliefs.
Since we have defined random variables as functions of events, the 
independence statement can be written for random variables as 

\begin{equation}
p(X = x | Y = y) = p(X = x) 
\quad
\mbox{ for all 	}
x, y \in A \cap B.
\end{equation}

\noindent
It follows from independence that the probability of 
both $X = x$ and $Y = y$ having definite values is
given by

\begin{equation}
P(X = x, Y = y) = 
P(X = x)
P(Y = y),
\end{equation}

\subsection{Bernoulli distribution.}

A Bernoulli random variable takes two possible values.
It is customary to call them success and failure, even though
it is particular to a given problem what we define as success.

The probability mass function is given by the expression

\begin{equation}
p(X = x) = xp + (1 - x)q
\end{equation}

\noindent
for $x$ taking the values $1$ or $0$, success or failure.
It is easy to check that this is indeed a PMF,
its values are between $[0,1]$ and it is normalized to $1$,

\begin{equation}
\sum_{x = 0}^{1} p(X = x) = q + p = 1
\end{equation}

We will revisit the Bernoulli random variable multiple
times thourgh out these lectures.

\subsection{Binomial distribution.}

Let us consider the random variable $X$ defined by the sum of independent 
Bernoulli processes:

\begin{equation}
X = X_{1} + \ldots + X_{n},
\end{equation}

\noindent
that is, we wish to compute the distribution of successes in $n$ independent Bernoulli trials.

Previously, we considered the probability of a coin giving $k$ heads in $n$ trials.
This is a Binomial process if we interpret heads as success.
%If we consider the sequence:
%
%\begin{equation}
%0010101010111101011
%\end{equation}
%
%\noindent
%where $1$ is interpreted as success and $0$ as failure, we can see that for $n = 19$
%we have obtained $k = 11$ successes, since each of these $n$ Bernoulli trials 
%is independent we can multiply the probabilities leading us to:
%
%\begin{equation}
%p(0010101010111101011) = 
%p^{11} q^{8}.
%\end{equation}
For general $k$ and $n$ we obtain

\begin{equation}
p(X = k) = {n \choose k} p^{k} q^{n - k},
\quad
\mbox{with }
q = 1 - p.
\end{equation}

\noindent
This PMF is positive defined, always less than $1$, and properly
normalized as a consequence of the binomial identity.

An interesting fact of binomial distributions is that given two 
independent binomial random variables $X$ and $Y$, their sum
$Z = X + Y$ also follows a binomial distribution.
A particular obvious fact if we think about coin tosses:
asking for the number of heads in $n$ trials, and then proceed
to ask for the number of heads in subsequent $m$ trials should
be distributed as the number of heads in $n + m$ trials.
We can mathematically prove this fact if we consider marginalization:

\begin{align}
P(X + Y = k) &= 
\sum_{j = 1}^{m} P(X + Y = k | Y = j) P(Y = j) 
\nonumber \\
\nonumber \\
&= \sum_{j = 1}^{m} P(X = k - j | Y = j) P(Y = j) =
\sum_{j = 1}^{m} P(X = k - j) P(Y = j) 
\nonumber \\
\nonumber \\
&= \sum_{j = 1}^{m} {n \choose k - j} p^{k-j} q^{n - k + j} \times
{m \choose j} p^{j} q^{m - j}
\nonumber \\
\nonumber \\
&= 
\sum_{j = 1}^{m} {n \choose k - j} {m \choose j} p^{k} q^{n + m -k}.
\end{align}

\noindent
If we consider that

\begin{equation} \label{Wandermond}
\sum_{j = 1}^{m} 
{n \choose k - j}
{m \choose j }
= 
{m + n \choose k}
\end{equation}

\noindent
we obtain the binomial distribution again.
Equation \eqref{Wandermond} is known as the Vandermonde
after the first mathematician who considered it.
It can be understood if we consider different ways of picking
$k$ objects of $m + n$ in total.
We can directly pick $k$ out of $m + n$ or we can pick $j$ out of $m$ and 
$k - j$ out of $n$. If we sum over all possible $j$ picks, these
two quantities should match. This is the Vandermonde.

Finally we have proved what we wanted, if $X$ and $Y$ are independent
binomial random variables, their sum $Z = X + Y$ is also a binomial random variable:

\begin{equation}
P(X + Y = k) = 
{m + n \choose k} p^{k} q^{m + n - k}
\end{equation}

\subsection{Hypergeometric distribution.}

Let us consider the following problem:
In a $5$-hand card, what is the distribution of aces.
And let us denote the random variable of interest by $X$ which
parametrizes the number of aces in a $5$-card hand, that is,
$X \in \{0, 1, 2, 3, 4\}$.

A common mistake is to think that $X$ is a binomial random variable,
since $X$ can be thought as the sum of Bernoulli random variables 
that are $1$ if the ace of given type is in the hand.
This argument is wrong because these Bernoulli random variables are
\textit{not} independent. Even though the Bernoulli variables take values
$0$ or $1$, their distributions are dependent of each other, if we know that a given
ace is in the hand, the likelihood of other aces being present change.

\paragraph{Remark.}
It is very common to mix a random variable and its distribution.
One thinks that because two random variables are dependent, their values must change.
This is not true. The probability distribution is what actually changes.

\vspace{.5cm}
In the case where the Bernoulli trials are dependent, 
$X$ is the hypergeometric distribution.
We have $52\choose{5}$ possible hands and given that $X = k$,
we have ${4 \choose k}$ possible aces and ${48 \choose 5 - k}$
non-ace cards.
By symmetry every possible configuration is likely and therefore
we can apply the naive definition of probability:

\begin{equation}
p(X = k) = 
\frac{ {4 \choose k} {48 \choose 5 - k} }
{ {52 \choose 5} }
\end{equation}

\noindent
This is a valid PMF because the sum over $k$ gives $1$ if we apply the 
Vandermonde.

The main difference between hypergeometric and binomial distributions
is replacement. Every time we actually get an ace from the deck we alter the 
frequency of another ace happening. If the trials we binomial, this would not 
happen. But if we consider the fact that the number of cards goes to infinite,
and the number of aces -- number of successes -- goes to infinite too,
it is intuitively clear that the frequencies must be almost unaltered.
It is also very unlikely that if we sample a given ace, we resample it again
on a subsequent trial because the number of aces is very large.
Therefore on this limit it should be possible to show that the hypergeometric
and binomial distributions are the same.
This indeed happens, but we will have to show it later.

The general case of a hypergeometric distribution is given by

\begin{equation}
P(X = k) = 
\frac{ {w \choose k} {b \choose n - k} }
{ {w + b \choose n} }
\end{equation}

\noindent
where $w$ and $b$ are the respective successes and failures of Bernoulli variables.

\subsection{Expectations I.}

\paragraph{Expectation.} We define the expectation operator, denoted $\mathbf{E}$, by:

\begin{equation}
\mathbf{E} [ X ]  = \sum_{x} x p_{X} (x) .
\end{equation}

\noindent
It has a nice interpration if one thinks of probabilities in terms of frequencies.
We average what we expect to have, $x$, weighted by the frequency we expect it to occur $p_{X}(x)$.
This definition is particularly useful because it is linear:

\begin{equation}
\mathbf{E} \bigg[  \alpha \left( X + Y \right) + \beta \bigg] = 
\alpha \mathbf{E} [X] + \beta +
\alpha \mathbf{E} [Y] + \beta.
\end{equation}

\noindent
It is linear for both \textit{dependent} and \textit{independent} random variables\footnote{A very very very non-intuitive fact.}.
Let us show that $\mathbf{E} [ X + Y ] = \mathbf{E} [ X ] + \mathbf{E} [ Y ]$ for both dependent and independent $X$ and $Y$.
The proof that linearity under real constants $\alpha$ and $\beta$ is left for the reader.
From the definition

\begin{align}
\mathbf{E} [ X + Y ] &=
\sum_{x = 1}^{\infty} \sum_{j = 1}^{\infty} (x + y) p (X = x, Y = y)
\nonumber \\
\nonumber \\
&= \sum_{x = 1}^{\infty} \sum_{j = 1}^{\infty} x p (X = x, Y = y) 
+
\sum_{x = 1}^{\infty} \sum_{j = 1}^{\infty} y p (X = x, Y = y)
\nonumber \\
\nonumber \\
&= \sum_{x = 1}^{\infty}  x p (X = x) 
+
\sum_{j = 1}^{\infty} y p (Y = y)
\nonumber \\
\nonumber \\
&= \mathbf{E} [ X ] + \mathbf{E} [ Y ] 
\end{align}

\noindent
where on the third equality we have used marginalization.

It is not true that for a general function of a random variable $X$, $g(X)$, that
$
\mathbf{E} \left[ g(X) \right] = 
g \left( \mathbf{E} \left[ X \right] \right)
$, which can be interpreted as it is not wise to reason on the avarage.
However it is true that given a function of a random variable $X$, $g(X)$, we can compute the expectation
$\mathbf{E} [ g(X) ] $ using

\begin{equation} \label{LOTUS}
\mathbf{E} \left[ g(X) \right] = \sum_{x} g(x) p_{X} (x)
\end{equation}

\noindent
Equation (\ref{LOTUS}) is commonly known as the law of 
unconscious statistician (LOTUS): it has been widely used by 
students without realization that it has to be proved. 
\textcolor{red}{\textbf{TODO: prove it!}}
%Since this result is important, we will prove it.
%Let us fix $g(X) = Y$, then
%
%\begin{equation}
%\mathbf{E} \left[ Y \right] =
%\sum_{y} y \, p_{Y} (y).
%\end{equation}
%
%\noindent
%It is pretty clear that the problem is to compute $p_Y (y)$, but in some sense, it is 
%also clear that $Y$ is completely defined in terms of $X$.
%Consider
%
%\begin{equation}
%p_Y (y) = p_Y ( g(x) ),
%\end{equation}
%
%\noindent
%such that
%
%\begin{equation}
%\mathbf{E} [ g(X) ] =
%\sum_{x;\, y = g(x)  }
%\end{equation}
%The expectation of a constant random variable is the constant itself: $\mathbf{E} [ A ] = a$ if the 
%random variable $A$ only takes the value of $a$ with probability $1$.
%Consider then $\mathbf{E} [ X - \mathbf{E} [ X ] ]$. Using past considerations, we conclude 
%that $\mathbf{E} [ X - \mathbf{E} [ X ] ] = 0$ and that the operator satisfies $\mathbf{E}^{2} = \mathbf{E}$.
%This result,
%
%\begin{equation}
%\mathbf{E} [ X - \mathbf{E} [ X ] ] = 0
%\end{equation}
%
%\noindent
%can be interpreted if we think of probabilities as frequencies: half of the time we will be above the avarage and the other half of the time we
%will be below avarage.
%The average collapses to zero.
%If we wish to know how far off from the mean we are irrespective of the sign, we essentially have to compute the distance from the mean.

Let us compute the expectations for the previously defined random variables.

\paragraph{Bernoulli.}
The expectation of a Bernoulli random variable $X$ is $\mathbf{E} [ X ] = p$.
This very simple statement can be used to generalize Bernoulli variables.
Let us define an indicator random variable $I$ such that $I = 1$ if
event $A$ happens. It follows that $\mathbf{E} [ I ] = p(A)$.

I do not know if this fact has a name, but I will call it a \textit{fundamental bridge}
because of its is of fundamental importance.
Expectations \textit{are} probabilities. One can understand and reformulate all
of probability theory using expectations. And indicator random variables
are very useful computational devices.

\paragraph{Binomial.}
Let us denote by $\sim$ the distribution of a random variable.
If $X$ has a binomial distribution, we write $X \sim \text{Binomial}[n,p]$.
Given that a binomial random variable is a sum of Bernoulli trials, by linearity
we can write

\begin{equation}
\mathbf{E} [ X ] = n \mathbf{E} [ X_{1} ]  = np .
\end{equation} 

To illustrate linearity it is interesting to compute this result
in another way. Consider from the definition

\begin{align}
\mathbf{E} [ X ] &=
\sum_{k = 0}^{n} k \, p(X = k) 
\nonumber \\
\nonumber \\
&= \sum_{k = 1}^{n} k \, p(X = k) 
\nonumber \\
\nonumber \\
&= \sum_{k = 1}^{n} 
k { n \choose k }  p^{k} q^{n - k}
\nonumber \\
\nonumber \\
&= \sum_{k = 1}^{n}
n {n - 1 \choose k - 1}
p^{k} q^{n - k}
\nonumber \\
\nonumber \\
&= n \sum_{k' = 0}^{n}
{n - 1 \choose k' }
p^{k' + 1} q^{n - k' - 1}
\nonumber \\
\nonumber \\
&= np 
\sum_{k' = 0}^{n}
{n - 1 \choose k' }
p^{k'} q^{n - k' - 1} = np
\end{align}

\paragraph{Hypergeometric.}
Let us consider the indicator random variables $X_{j}$ of the $j$th card
in a $5$-hand card being an ace, $j \in \{1,2,3,4,5\}$.
By linearity we obtain

\begin{equation}
\mathbf{E} [ X_{1} + \ldots + X_{5} ] =
5 \mathbf{E} [ X_{1} ] = 5 \times \frac{4}{52}.
\end{equation}

\noindent
where on the last line we used \textit{the fundamental bridge} between
expectations and probabilities that $\mathbf{E} [X_{1}] =
p( \text{first card being an ace} ) = 4 / 52$.

For a general hypergeometric, the sum of Bernoulli random variables is dependent.
But expectations are linear for dependent random variables as well.
We directly conclude that

\begin{equation}
\mathbf{E} [ X ] = 
n p = n \left( \frac{w}{w + b} \right).
\end{equation}

\noindent
For the special case of the aces we obtain $p = 4/52$.
The hypergeometric distribution is particularly useful if one
wishes to understand probabilities of poker.

\paragraph{Putnam problem.}
Consider a random permutation of $\{1, 2, \ldots, N\}$ with $N \geq 2$.
We say that we have a local maxima if the value is greater than its neighbours,
for example in the permutation

\begin{equation}
\{1, 2, 3, 4, 5, 7, 6\}
\end{equation}

\noindent
we say that $7$ is a local maxima since it is greater than its two neighbours,
$5$ and $6$.
Find the expectation of local maxima.

\paragraph{Answer.}
This problem was considered the most difficult question in the putnam exam, but
we can solve it in a few lines using the concept of 
indicator random variables.

Let $X_j$ be the indicator of position $j$ having a local maxima, and 
$X = X_{1} + \ldots + X_{N}$ being the random variable that gives the number of
local maxima. The boundary variables $I_{1}$ and $I_{N}$ can only be greater
than $X_{2}$ and $X_{N-1}$; since each position is equally likely, the probability
that $X_{1} > X_{2}$ is $1/2$, and so

\begin{align}
\mathbf{E} [ X ] &= 
\mathbf{E} [ X_{1} ] + \mathbf{E} [ X_{N} ] + 
\mathbf{E} [ X_{2} + \ldots + X_{N - 1} ]
\nonumber \\
\nonumber \\
&= \frac{1}{2} + \frac{1}{2} + 
\mathbf{E} [ X_{2} + \ldots + X_{N - 1} ]
\nonumber \\
\nonumber \\
&= 1 + 
\mathbf{E} [ X_{2} + \ldots + X_{N - 1} ]
\nonumber \\
\nonumber \\
&= 1 + (N - 2) \mathbf{E} [ X_{2} ] = 1 + (N - 2) p(X_{2}).
\end{align}

\noindent
All that is left to do is to compute the probability 
that position $X_{2}$ is a local maxima.
By symmetry, any of the three positions is equally likely 
to have the maximum number, which gives, using the naive definition, 
a probability of $1/3$; the local maxima has lend right in the middle.
Therefore

\begin{equation}
\mathbf{E} [ X ] =
1 + \frac{(N - 2)}{3}	.
\end{equation}

A common mistake is to think that the middle position has 
$1/2$ chance of being greater than the number on the left, and a
$1/2$ chance of being greater than the number on the right, giving
a total of $1/4$. Unfortunately, $X_{j}$ are not independent
random variables, if we know that the number on the right is 
lower than the number on the left, and the number on the left is 
lower than the central number, it must follow that the number on 
the right is lower than the central number.

\subsection{Geometric Distribution.}
Let us consider independent Bernoulli trials and count the number of 
failures until the first success. This is a geometric process
with probability mass function given by

\begin{equation} \label{geometricdistribution}
P(X = k) = p q^{k}.
\end{equation}

\noindent
It is called geometric distribution because it is given by a geometric series:

\begin{equation}
\sum_{k=0}^{\infty} p q^{k} = \frac{p}{1 - q} = 1.
\end{equation}

\paragraph{Memory.}
Let us show that the geometric distribution is a memoryless process.
Suppose now that the first $t$ tosses are tails, we wish 
to calculate 
$ p ( X \geq t + s \, | \, X \geq t ) $, that is
the probability that the first head happens at toss $t + s$ given 
that we have already failed at least $t$ times.
If this is indeed a memoryless process, this must be equal to

\begin{equation} \label{memorylessProcess}
p ( X \geq t + s \, | \, X \geq t ) = 
p ( X \geq s ).
\end{equation}

To prove this statement we apply the definition of conditional probability:

\begin{align}
p ( X \geq t + s \, | \, X \geq t ) = 
\frac{p ( X \geq t + s \, \, \text{and} \, \, X \geq t )}{ p( X \geq t) } =
\frac{p ( X \geq t + s )}{ p( X \geq t) } 
\end{align}

\noindent
and observe that the event $X \geq t$ is inside the event $X > t + s$ and so 
the intersection is trivial.
If 

\begin{equation}
\frac{p ( X \geq t + s )}{ p( X \geq t) }  = 
p ( X \geq s )
\end{equation}

\noindent
is true, then it must also be true that

\begin{equation} \label{geometricmemorylesstoprove}
p( X \geq t + s ) =
p( X \geq s ) p ( X \geq t).
\end{equation}

\noindent
And we are going to prove \eqref{geometricmemorylesstoprove} now.
From the definitions:

\begin{equation}
p( X \geq t + s ) = 1 - p( X < t + s )
\end{equation}

\noindent
and we have to calculate the probability of failing \textit{for less than} $t + s$ times,
considering that we could fail at the first toss, or the second, or the third and so on,
the probability is given by

\begin{equation}
P( X < t + s ) = 
p + pq + pq^{2} + \ldots + p q^{t + s - 1}
\end{equation}

\noindent
since all of these events are disjoint, you cannot fail at the first toss and the second toss.
This is a partial geometric series and we can write its the sum as:

\begin{equation}
P( X < t + s ) =
1 - q^{t + s}.
\end{equation}

\noindent
This result can be recycled for the other too computations so that we obtain

\begin{equation}
P( X \geq t + s ) =
q^{t + s},
\quad
P( X \geq s ) = 
q^{s}
\quad
\text{and}
\quad
P( X \geq t ) =
q^{t}
\end{equation}

\noindent
and memorylessness is satisfied.
It is possible to show that the geometric distribution is the only
memoryless distribution. \textcolor{red}{(TODO)}

\paragraph{Expectation.}
We can use the memorylessness property to compute the expected value
of the geometric distribution. Remember that we have defined 
the random variable $X$ to be the number of failures until the first success,
which means that if we have obtained a success on the first trial, we have zero failures,
$0 \times p $. However, if we fail on the first trial, because the process is memoryless,
we have to start it all over again; mathematically:




leading us to the following equation

\begin{equation}
\mathbf{E} [ X ] = 0 p + ( 1 + \mathbf{E} [X] ) q
\end{equation}

\noindent
which can be solved directly

\begin{equation}
\mathbf{E} [ X ] = \frac{q}{p}.
\end{equation}

We could have computed the expectation from the definition

\begin{align}
\mathbf{E} [ X ] &=
\sum_{k = 1}^{\infty}
k p q^{k} = 
p \sum_{k = 1}^{\infty} k q^{k}
\end{align}

\noindent
with help of

\begin{equation}
q \frac{\partial}{\partial q}
\sum_{k = 0}^{\infty} q^{k} =
q \frac{\partial}{\partial q}
\left( \frac{1}{1 - q} \right) =
\frac{q}{(1 - q)^{2}}
\end{equation}

\noindent
we again obtain the same result

\begin{equation}
\mathbf{E} [X] =
\frac{q}{p}.
\end{equation}

\paragraph{Example.}
Let us count the number of tosses until first success instead of the number of 
failures. If I have $k$ failures until the first success, it means I have tossed the coin $k + 1$ 
times. Let this random variable be represented by $X$; it is related to the geometric distribution 
$Y$ via $X = Y + 1$, which means 

\begin{equation}
X - 1 \sim \text{Geometric} (p).
\end{equation}

\noindent
The geometric distribution \eqref{geometricdistribution} implies

\begin{align}
P(X - 1 = k) = p q^{k}
\quad
\Rightarrow
\quad
P(X = k + 1) = p q^{k}.
\end{align}

\noindent
And we can redefine $k + 1 = k'$ to count powers of $X$ from which we obtain

\begin{equation}
P(X = k') = p q^{k' - 1}.
\end{equation}

\noindent
Notice that this distribution is already normalized.
The expectation of this distribution can be calculated directly from the
expectation of the geometric and linearity:

\begin{equation}
\mathbf{E} [X] = \mathbf{E} [Y] + 1 = \frac{q}{p} + 1 = 
\frac{1}{p}.
\end{equation}

\noindent
This expectation just states the intuitive fact that we have to wait
a period of $1/p$ if we have a frequency of $p$ for a result to happen.
Considering a fair coin, this means that on average we need $2$ tosses
to see the first heads since there is only two possibilities which are
equally likely.


\subsection{Negative Binomial distribution.}

We could generalize the story behind the geometric distribution and consider the 
random variable $X$ defined as the number of failures until we reach $r$ successes in independent Bernoulli
trials.
An example would be the sequence:

\begin{equation}
1000100100001001.
\end{equation}

\noindent
In this example $r = 5$ in $n=11$ trials.
Notice that whatever the sequence turns out to be, it must always end with a $1$. 
We do not care about the positions of our successes in the past $10$ trails,
therefore

\begin{equation}
P( X = n ) = { n + r - 1 \choose n } p^{r} q^{n},
\end{equation}

\noindent
for $n = 0, 1, 2, \ldots$.

\paragraph{Expectation.}
The expectation of this negative binomial distribution can be calculated directly
with the help of indicator random variables. 
Let $I_j$ be the random variable that measures the number of failures between the 
success $j-1$ and $j$, that is 

\begin{equation}
I_{j} \sim \text{Geometric}(p)
\end{equation}

\noindent
and then 

\begin{equation}
\mathbf{E}[X] = \mathbf{E} \left[ 
I_1 + \ldots + I_r \right] = r \left( \frac{q}{p} \right).
\end{equation}

\subsection{Poisson distribution.}

In this section, we will define the Poisson distribution,
which is arguably the most useful discrete random variable.
It can model any process with very large sample space, where 
each subset of the sample space has a very tiny probability of sucess.

Let us consider the Taylor series for $e^{\lambda}$:

\begin{equation}
\exp{\lambda} = \sum_{k = 1}^{\infty} \frac{\lambda^{k}}{k !}
\end{equation}

\noindent
and notice that it gives a normalized probability mass function

\begin{equation}
1 = \sum_{k = 1}^{\infty} e^{-\lambda}  \frac{\lambda^{k}}{k !},
\end{equation}

\noindent
for the Poisson process:
	
\begin{equation}
P(X = k) = e^{-\lambda} \frac{\lambda^{k} }{k!}
\end{equation}

\noindent
where $\lambda$ is any positive real number, and $k \in \{0, 1, 2, \ldots \}$.

\paragraph{Poisson approximation.}
Take $A_{1}, \ldots, A_{n}$ events with $n$ large such that
$p(A_{1}) = p_{1}, \ldots, p(A_{n}) = p_{n}$ are very small.
If $A_{1}, \ldots, A_{n}$  are independent then the number of 
$A_{j}$'s that occur is approximately Poisson with 
$\lambda = \sum_{j = 1}^{n} p_{j}$.

\vspace{.5cm}
The Poisson approximation formalizes our intuition that a Poisson process
models very well a large sample space where each event has a tiny probability.
It even gives an estimation for $\lambda$.
However, it does not provide an estimate for $n$. The approximation actually works 
very well even if $n$ is not large.

A consequence of the Poisson approximation is that an infinity number of independent
Bernoulli trials is approximated by the Poisson distribution if the probability of 
each Bernoulli trial is small.
Mathematically, the Poisson distribution must result from the limit $n \to \infty$ 
together $p \to 0$ applied on the binomial distribution.
Let $X \sim \text{Binomial[n, p]}$ with $n \to \infty$ and $p \to 0$ keeping
$np$ fixed. Define $\lambda = np$ such that

\begin{align}
P(X = k) &= 
{n \choose k}
p^{k} q^{n - k}
= {n \choose k} \left( \frac{\lambda}{n} \right)^{k}
\left( 1 - \frac{\lambda}{n} \right)^{n - k}
\nonumber \\
\nonumber \\
&= \frac{n!}{k! (n - k)!} 
\left( \frac{\lambda}{n} \right)^{k}
\left( 1 - \frac{\lambda}{n} \right)^{n - k}
\nonumber \\
\nonumber \\
&= \frac{n!}{n^{k} (n - k)!} 
\left( \frac{\lambda^{k}}{k!} \right)
\left( 1 - \frac{\lambda}{n} \right)^{n - k}
\nonumber \\
\nonumber \\
&= \frac{n \times (n-1) \times \ldots \times (n - k + 1)}{n \times n \times n \ldots \times n} 
\left( \frac{\lambda^{k}}{k!} \right)
\left( 1 - \frac{\lambda}{n} \right)^{n - k}.
\end{align}

\noindent
Taking the limit $n \to \infty$ the first term goes to $1$, while the last
term gives the definition of exponential function, and finally

\begin{equation}
\lim_{n \to \infty} P_{\tiny{Binomial}} (X = k ) = 
e^{- \lambda} \frac{\lambda^{k}}{k!}
\end{equation}

\paragraph{Example.}
Find approximation for the probability that there are $3$ people with the same 
birthday.

\paragraph{Answer.}
There are ${n \choose 3}$ groups of people with $3$ people. 
Let $I_{ijk}$ with $i < j < k$ be the indicator random variable that 
$3$ people have the same birthday, therefore

\begin{equation}
\mathbf{E} [ X ] =
{n \choose 3} \mathbf{E} [ I_{123} ] =
{n \choose 3}
\frac{365}{365} \frac{1}{365} \frac{1}{365}
\end{equation}

\noindent
is the \textit{exact} expectation that $3$ people have the same birthday.
Now consider that the number of possible groups with $3$ people 
is very large, but it is very unlikely that a group contains $3$ people
that have birthdays at the exact same date, hence the Poisson approximation
is valid and we can write

\begin{equation}
P(X \geq 1) = 1 - P(X = 0) = 1 - \frac{\lambda^{0}}{0!} e^{-\lambda} = 
1 - e^{-\lambda}
\end{equation}

\noindent
with $\lambda$ being equal to $\mathbf{E}[X]$.

The expected value for the Poisson distribution is 

\begin{align}
\mathbf{E} [ X ] &= \sum_{k=0}^{\infty} k P(X = k)
\nonumber \\
\nonumber \\
&= e^{-\lambda} \sum_{k = 1}^{\infty} k \frac{\lambda^{k}}{k!}
\nonumber \\
\nonumber \\
&= e^{-\lambda} \sum_{k = 1}^{\infty} \frac{\lambda^{k}}{(k - 1)!}
\nonumber \\
\nonumber \\
&= \lambda e^{-\lambda} \sum_{k = 1}^{\infty} \frac{\lambda^{k-1}}{(k - 1)!}
\nonumber \\
\nonumber \\
&= \lambda e^{-\lambda} e^{\lambda} = \lambda
\end{align}

\section{Solved Problems.}

\subsection{Gambler's ruin.}

\paragraph{Problem.}
Two gamblers, $A$ and $B$, play a sequence of bets such that
$P(A\text{ wins a round}) = p$. 
What is the probability that $A$ wins the entire game.
Assume that $A$ starts with $i$ and $B$ with $N-i$.

\paragraph{Answer.}
We have, from the point of view $A$, a random walk.
Since each round is independent, $A$ has the same probability of winning each round.
This allows us to generate a recursive equation for the 
probability that $A$ wins the game.
Let us define

\begin{equation}
p_{i} = \text{Probability of A winning the game starting with } i \text{ dollars.}
\end{equation}

\noindent
and then condition on the first round:

\begin{align}
p_{i} &= p p( \text{A winning starting with } i \, | \, \text{A won the first round} ) 
\nonumber \\
&+ q p( \text{A winning starting with } i \, | \, \text{A lost the first round} ).
\end{align}

\noindent
but if $A$ won the round, since each trial is independent is as if the game starts again, but 
now $A$ has $i + 1$ dollars if he/she has won or $i - 1$ dollar if he/she has lost.
Therefore we obtain

\begin{equation}
p_{i} = p p_{i + 1} + q p_{i - 1}.
\end{equation}

\noindent
This type of difference equation is subjected to the following boundary conditions:
$p_{0} = 0$ and $p_{N} = 1$, that is, $A$ has no chance of winning if has no more money left,
and $A$ surely wins if he/she has all available money in the game.
The solution is given by

\begin{equation}
p_{i} = 
\begin{cases}
\frac{1 - (q/p)^{i}}{ 1 - (q/p)^{N}} &\mbox{if } q \neq p
\\
\frac{i}{N} &\mbox{if } q = p = 1/2.
\end{cases}
\end{equation}

\noindent
I have computed some probabilities.
If the game has a slightly disadvantage $p = 0.49$, and we start with half of the money, we obtain
$p_{0.5} = 0.49$ as a probability of winning.
However, if with the same odds $p=0.49$ we start with $i = 0.7$ of the money we obtain
$p_{i} = 0.69$. It is interesting that our probability of winning increases if we have more money,
and unintuitive. I think it just reflects the fact that you might be able to afford more losses
if you have more money even if the odds are against you.

\subsection{Cupom Collector problem.}

\paragraph{Problem.}
There are $n$ toy types given by a Fast Food restaurant that are equally likely to be obtained.
What is the expected time until we have the complete set of toys?

\paragraph{Answer.}
Let us define the random variable $T_{n}$ as the time to get a new toy, that is, $T_{1}$ is the time
until first new toy, $T_{2}$ is the time until the second new toy, \textit{etc.}
The probability of obtaining the first toy is $1$, since you always start with a new toy;
it follows that the second toy, out of $n$ has a probability $(n-1)/n$ of occuring. The 
same reasoning can be applied to the probability of obtaining all $n$ toys.

Each $T_{i}$ is then distributed as geometric random variable:

\begin{align}
&T_{1} - 1 \sim \text{Geometric}[1]
\nonumber \\
\nonumber \\
&T_{2} - 1 \sim \text{Geometric}\left[ \frac{n - 1}{n} \right]
\nonumber \\
\nonumber \\
&T_{3} - 1 \sim \text{Geometric}\left[ \frac{n - 2}{n} \right]
\nonumber \\
&\quad \vdots
\nonumber \\
&T_{n} - 1 \sim \text{Geometric}\left[ \frac{1}{n} \right]
\end{align}

The expectation can be computed via linearity:

\begin{align}
\mathbf{E} \left[ T_{1} + \ldots + T_{n} \right] 
&= \frac{n}{n-1} + \ldots + \frac{n}{1}
\nonumber \\
\nonumber \\
&= n \bigg( 1 + \frac{1}{2} + \ldots + \frac{1}{n} \bigg)
\nonumber \\
\nonumber \\
&\sim n \log n .
\end{align}

\noindent
In the last line we have used the approximation for the harmonic series.

\subsection{Blissvile and Blotchville problem.}

\paragraph{Problem.}
Fred lives in Blissville where buses always arrive on time with the time difference
between successive buses fixed at $10$ minutes.
Having lost his watch, he arrives at the bus stop at a uniformly random time on a certain day
(assume that buses run $24$ hours a day, every day, and that the time Fred arrives is 
independent of the bus arrival process).

\begin{enumerate}
\item What is the distribution of how long Fred has to wait for the next bus?
What is the average time Fred has to wait?

\item Given that he bus has not yet arrived after $6$ minutes, what is the probability 
that Fred will have to wait at least $3$ more minutes?

\item Fred moves to Blotchville, a city with inferior urban planning and where buses
are much more erratic. Now, when any bus arrives, the time until the next bus arrives is an Exponential
random variable with mean $10$ minutes.
Fred arrives at the bus stop at a random time, not knowing how long ago the previous bus came.
What is the distribution of Fred's waiting time for the next bus? What is the average time 
that Fred has to wait?

\item When Fred complains to a friend how much worse transportation is in Blotchville, the friend
says: \textit{Stop whinning! You arrive at a uniform instant between the previous bus arrival and 
the next bus arrival. The average length of that interval between buses is $10$ minutes, but
since you are equally likely to arrive at any time in that interval, your average waiting time is 
only $5$ minutes}. Why the friend's reasoning is wrong?
\end{enumerate}

\paragraph{Answer.}
\begin{enumerate}
\item As soon as Fred arrives at the bus stop, 
the bus can arrive at any time between his arrival and $10$ minutes with
equal probability. The waiting time distribution is uniform on the interval $[0,10]$;
and therefore the mean is $5$.

\item We have to compute

\begin{equation}
P( t > 3 + 6 \,|\, t \geq 6 ) = 
\frac{ P( t > 3 + 6 ) }{ P( t \geq 6 ) } =
\frac{1/10}{4/10} = 
1 / 4
\end{equation}

\item Because the exponential distribution is a memoryless process, after Fred arrives at the bus 
station, the waiting time distribution is also exponential with mean $10$ minutes.
The average waiting time is $10$ minutes.
The previous problem is not memoryless, the waiting time is for the next $3$ minutes has probability:
$P( t > 3 ) = 7/10$, which is different from $1/4$.

\item This is a case of length-time bias. The friend is mistaking the average time interval between
bus arrivals for the actual time of bus arrivals.
In reality, Fred is more likely to arrive during long interval times than on short intervals.
\end{enumerate}

\subsection{Expected number of Heads Heads -- Brute force.}

\paragraph{Problem.}
If we consider a fair coin, what is the expected number of tosses
until we obtain Heads Heads for the first time? And Heads Tails?

\paragraph{Answer.}
Later we will find explore different ways to answer this question.
For the moment, let us approach this problem in a very brute force way:
counting how many sequences of length $n$ that do not contain $HH$ inside
exist.

We will count the number of sequences in a recursive way.
Let us denote the sequences of length $n$ that contain no $HH$ by
$g_{n}$.
And first, let us count the number of sequences for $n=1$:

\begin{equation}
\begin{pmatrix}
H \\
T
\end{pmatrix}.
\end{equation}

\noindent
Each possible sequence is a line in this matrix, and we have $2$ sequences in total.
To obtain $n = 2$, we observe that it is always possible to attach
a tails $T$ to any of the previous sequences, giving a total of $ 1 \times g_{n - 1}$.
However, we are allowed to add a heads $H$ if and only if the previous sequence
does not end in heads. How many of these sequences that end with $T$ are there?
There are $g_{n - 2}$ because for every sequence of length $n-2$ we can add a tails 
in the end by our previous consideration.
Recursively, we can generate $g_{n}$:

\begin{equation} \label{fibo}
g_{n} = g_{n-1} + g_{n-2}
\end{equation}

\noindent
provided that we know the values of $g_{2}$ and $g_{1}$.
This recursion is valid for $n \geq 3$. The value of $g_{1}$
has already been computed, $2$. For $g_{2}$ we obtain:

\begin{equation}
\begin{pmatrix}
H T \\
T T \\
T H
\end{pmatrix},
\end{equation}

\noindent
a total of $3$ sequences.
Equation \eqref{fibo} subjected to conditions $g_1 = 2$ and $g_2 = 3$ 
is the Fibonacci sequence.

The sample space is the space of all sequences of $H$ and $T$ of length
$n$. Since each spot of the sequence has $2$ possibilities, we obtain
a $2^{n}$ number of possible configurations.
Therefore, the probability that a sequence of length $n$ ends with
$HH$ is

\begin{equation}
p_n = \frac{g_{n - 3}}{2^{n}}.
\end{equation}

\noindent
The factor $g_{n - 3}$ enters in the expression because a sequence that ends with
$HH$ must be a sequence of length $n-2$ that ends with a tails $T$ -- something like
$... THH$ -- and there are $g_{n-3}$ sequences that contain no $HH$ and end with a 
$T$. This expression allows a recursive formula for the probability $p_{n}$:

\begin{equation} \label{recHH}
p_{n} = \frac{1}{2} p_{n - 1} + \frac{1}{4} p_{n - 2}
\end{equation}

\noindent
where we have just replaced the Fibonacci sequence for $g_{n-3}$.
The recursion \eqref{recHH} is valid for $n \geq 4$ and has boundary 
conditions $p_{2} = 1/2$ and $p_{3} = 1/8$ -- the only sequence of
length $3$ that ends with $HH$ is $THH$.

The explicit solution is

\begin{equation}
p_{n} =  \frac{5 - \sqrt{5}}{10} \left( \frac{1 + \sqrt{5}}{4} \right)^{n} 
+ 
\frac{5 + \sqrt{5}}{10} \left( \frac{1 - \sqrt{5}}{4} \right)^{n} 
\end{equation}

\noindent
and the expected value can be computed

\begin{equation}
\mathbf{E} [ HH ] = \sum_{n = 2}^{\infty} n p_{n} = 6
\end{equation}

\noindent
which means that on average we have to toss $6$ times until
we see the first $HH$.

The average number of tosses until $HT$ is a bit easier.
We first wait for the first time $H$ appears, and
then after the first $H$ has appeared we wait again
for the first $T$.
That is we expect $2$ tosses until $H$ and $2$ more tosses
until $T$: 

\begin{equation}
\mathbf{E} [ HT ] = 2 + 2 = 4
\end{equation}

These two numbers are different. 
And at first glance this is very unintuitive since we think
that because $H$ and $T$ occur with the same probability
there is a symmetry which dictates that every possible sequence
is equally likely and thus every pattern should occur with equal
expectation.

This reasoning is almost entirely correct. Indeed there is a symmetry
between $H$ and $T$, but the symmetry actually simply implies that if 
we substitute in every expression $H$ by $T$ and vice-versa, we must
obtain the same probabilities. This gives $\mathbf{E} [ HH ] = \mathbf{E} [ TT ]$.
It is also true that every sequence should occur with equal probability, since
on average the appearance of $H$ has exactly the same frequency of $T$.

\begin{center}
\textcolor{red}{TODO: RIGHT MORE ABOUT THIS}
\end{center}


\section{Continuous random variables.}

The main difference between continuous and discrete random variables is that 
continuous random variables can take values on an interval in the real line whereas 
discrete random variables can take an enumarate set of values.

A particular stricking consequence of this definition is that the PMF of 
a continuous random variable is zero: $P(X  = x) = 0$.
We can see this if we think about the relationship between the PMF and the CDF;
a well-defined CDF converges to $1$ and has jumps each time a probability of an 
event is different from zero. The height of the jump is by definition the value of 
the PMF. A continuous random variable on the other hand has a continuous CDF since it 
can take any value on the real line, which implies the total probability increases
smoothly as we travel further in the line. 
The CDF of a continuous random variable has no jumps and therefore has
zero PMF.

Interstingly enough, a continuous CDF has a derivative.
Something that it is not true in the discrete case because of the jumps.
Let us call the derivative of a continuous CDF the probability density function (PDF)
usually denoted $\rho(x)$.
By the fundamental theorem of calculus, it is valid for any continuous random variable
$X$:

\begin{equation}
P( X \in B) = \int_{B} \rho(x) \mathrm{d}x
\end{equation}

\noindent
where $B$ is any subset of the real line, $\rho(x)$ is the PDF of $X$.
Moreover, since each individual point has zero probability, the
following equalities are true:

\begin{equation}
P( a \leq X \leq b ) = P( a < X < b ) = P( a < X \leq b ) = P ( a \leq X < b).
\end{equation}

In general, a valid PDF can be any function $\rho(x)$ that satisfies two properties:

\begin{itemize}
\item Non-negative: $\rho (x) > 0$ for every $x \in B$
\item Its integral over $\mathbf{R}$ exists and gives $1$: $\int_{-\infty}^{+\infty} \rho(x) = 1$
\end{itemize}

\subsection{Uniform distribution.}

A uniform random variable has probability density defined by

\begin{equation}
\rho(x) = 
\begin{cases}
\frac{1}{(b - a)}, \quad &\text{for } x \in [a,b], \\
\, 0, \quad &\text{otherwise.}
\end{cases}
\end{equation}

\noindent
The probability is equally distributed along the interval $[a,b]$.
The cumulative probability function can be directly computed

\begin{equation}
F(x) = \int_{a}^{x} \frac{\mathrm{d} x}{(b - a)}  
= \frac{x - a}{b - a}
\end{equation}

\paragraph{Universality of the uniform.}
The uniform distribution is very useful for simulation of random variables.
Consider that we have $X$ with CDF $F(x)$. It follows if $U \sim \text{Uniform}(0,1)$
that $F^{-1}(U) = X$.

The proof of this statement is a direct computation 
of the CDF of $F^{-1}(U)$ (Notice that $F^{-1}(U)$ takes a number between $[0,1]$ to
the range of $x$):

\begin{align}
P(F^{-1} ( U ) \leq x )  &= 
P(U \leq F(x) ) = F(x)
\end{align}

\noindent
where the last line follows because the CDF of a uniform on $[0,1]$ is 
proportional to the interval, which is an interval defined by $[0, F(x)] \subset [0, 1]$.

Possible complications can arrise depending on the nature of $F(x)$. 
For instance, it could be non-invertible.
Most of the time, we will not be worried about these things, because a real
function can always be inverted on some subset of the original domain, the 
inverse function theorem in real analysis.

Let us apply this theorem on an example. 
If we consider the exponential distribution:

\begin{equation}
F(x) = 1 - e^{- x}
\end{equation}

\noindent
and set $u = F(x)$, we obtain

\begin{equation}
\ln \left( 1 - u \right) = x.
\end{equation}

\noindent
Distribution-wise $1 - U \sim U$ and so we can ignore the constant inside the logarithm.
By the universality of the uniform, we obtain that

\begin{equation}
X = \ln U
\end{equation}

\noindent
is distributed with CDF $F(x)$.

\subsection{Normal distribution.}

The most famous distribution of all statistics, the normal distribution is extremely used
because the central limit theorem\footnote{To be defined later.}.

\paragraph{Standard Normal distribution.}
A continuous random variable $Z$ is a standard normal, $Z \sim N(0,1)$, if
its PDF is given by

\begin{equation}
\rho(x) = \frac{1}{\sqrt{2 \pi}} e^{-z^{2}/2}, 
\quad
\text{where }
z \in \mathbf{R}.
\end{equation}

\noindent
The CDF of a normal distribution is usually denoted $\Phi (z)$, and it does not 
have a closed form using elementary functions:

\begin{equation}
\Phi(z) = \int_{-\infty}^{z} \frac{1}{\sqrt{2 \pi}} e^{-t^{2}/2} \mathrm{d} t.
\end{equation}

\noindent
There are several symmetry properties for the normal distribution:

\begin{itemize}
\item The PDF $\rho(z)$ is an even function.
\item Because the PDF is even, the CDF satisfies: $\Phi(-z) = 1 - \Phi(z)$.
To prove this, use the substitution $z \mapsto -z$.
\item From the previous considerations it follows that $-Z \sim N(0,1)$ also.
Just compute $P(-Z \leq z)$ and apply the previous property.
\end{itemize}

\paragraph{Normalization constant.}
Even though the CDF function cannot be computed. Its value at infinity can be calculated, $\Phi ( \infty ) $.
The trick is to square the integral, which allows conversion to polar coordinates:

\begin{equation}
\int_{-\infty}^{+\infty} 
\int_{-\infty}^{+\infty} 
e^{- (x^{2} + y^{2}) / 2}
\mathrm{d}x \mathrm{d} y
= 
\int_{0}^{2 \pi} 
\int_{0}^{+\infty} 
e^{- r^{2} / 2}
r \mathrm{d}r \mathrm{d} \theta
= 
2 \pi.
\end{equation}

\paragraph{Location-scale transformation.}
Let $X$ be a random variable and $Y = \sigma X + \mu$ where
$\sigma$ and $\mu$ are two real numbers with $\sigma > 0$.
We say that $Y$ is obtained by a location-scale transformation of $X$.
For example, if $X \sim \text{Uniform}[a,b]$ and $Y = c X + d$ then
$Y \sim \text{Uniform}[ca + d, cb + d]$.
Beware: the location-scale transformation
should be applied to the random variable itself and not to its 
PDF; the confusion is a case of sympathetic magic.

\paragraph{Normal distribution.}
If $Z \sim N(0,1)$ then $X = \mu + \sigma Z$ is said to have
normal distribution $X \sim N(\mu, \sigma^{2})$.
We can get back from $X$ to $Z$ via \textit{standarlization}.
Define

\begin{equation}
\frac{X - \mu}{\sigma^{2}} \sim N(0 , 1)
\end{equation}

\noindent
for $X \sim N(\mu, \sigma^{2})$.
This process allows us to extend all of our results for standard normal to a normal distribution.
The CDF becomes

\begin{subequations}
\begin{equation}
F(x) = \Phi \bigg( \frac{x - \mu}{\sigma} \bigg)
\end{equation}

\noindent
and the PDF can be obtained by taking the derivative

\begin{equation}
\rho(x) = \frac{1}{\sigma} \rho_{\texttt{SN}} \bigg( \frac{x - \mu}{\sigma} \bigg)
\end{equation}
\end{subequations}

\noindent
where $\texttt{SN}$ denotes standard normal.

\paragraph{Benchmarks of normal distribution.}
Let $Z \sim N(\mu, \sigma^{2})$, then it is true that

\begin{center}
\begin{itemize}
\item $P ( |Z| < 1 ) \approx 0.68$;
\item $P ( |Z| < 2 ) \approx 0.95$;
\item $P ( |Z| < 3 ) \approx 0.997$.
\end{itemize}
\end{center}

\subsection{Exponential distribution.}

If we consider a geometric process in continuous time where where 
succeses arrive at a rate $\lambda$ per unit of time we obtain an exponential
distribution. This is a very interesting result that we shall discuss in a second.
But first, let us introduce the exponential distribution.

% Look at exercise 45 of Blietzstein.
%Let the time interval be sliced in segments of length $\Delta t$.
%For a process that starts at time $0$ and ends at time $T$, we have defined a 
%partition on the interval $T - 0 = n \Delta t$,
%where $n$ is the number of intervals $\Delta t$ that fit inside $T - 0$.
%In the limit $n \to \infty$, we obtain a continuous time. 
%The probability of success $p$ on each interval $\Delta t$ is a constant 
%$\lambda$ such that in the total interval we expected a constant rate of success:
%$p = \lambda \Delta t$.
%Consider then
%\begin{equation}
%p ( 1 - p )^{k} 
%= \lambda \Delta t  \left(  1 - \lambda \Delta t \right)
%\end{equation}

The PDF of an exponential distribution is
given by

\begin{equation}
\rho(x) = \lambda e^{-\lambda x}
\quad 
\text{for } 
x \in [0, \infty)
\end{equation}

\noindent
and we say that $X \sim \text{Exponential}[\lambda]$. 
The cumulative distribution is 

\begin{equation}
F(x) = \int_{0}^{x} \lambda e^{- \lambda u} \mathrm{d} u =
1 - e^{-\lambda x}.
\end{equation}

\noindent
The exponential distribution is not defined over the entire
real line, the location-scale transformation will alter its distribution,
precisely it will modify its range.
However, it is possible to use a scale transformation.
Let $X \sim \text{Exponential} [1] $ then define 
$Y = X / \lambda$; it follows that $Y \sim \text{Exponential} [ \lambda ] $.
Conversely, $\lambda Y \sim \text{Exponential}[1]$ if $ Y \sim \text{Exponential} [ \lambda ]$.

\paragraph{Memory.}
The exponential distribution is a memoryless process.
If you have waited days for your success, it hasn't become more likely.
One can also show that the exponential distribution is the only memoryless
distribution. If we recall equation \eqref{memorylessProcess}

\begin{equation} 
p ( X > t + s \, | \, X \geq t ) = 
p ( X > s )
\end{equation}

\noindent
which is the definition of a memoryless process,
let us define $G(t) = p ( X > t )$ allowing us to rewrite
it as 

\begin{equation} \label{memoryless2}
G(t + s) = G(t) G(s). 
\end{equation}

\noindent
Some properties can be deduced from \eqref{memoryless2}.
First, take $t = s = 0$, giving $G(0) = G^{2}(0)$ which implies $G(0) = 1$\footnote{Or $0$, a solution that we will ignore.}.
Next, take $t = -s$ from which we find that $G(-t) = G^{-1}(t)$. 
Finally, differentiating with respect to $s$ and applying the result at $s = 0$ leads us to

\begin{equation}
\dot{G}(t) = \dot{G}(0) G(t) = \lambda G(t).
\end{equation}

\noindent
It is easy to verify that the exponential distribution is the 
solution of this differential equation given that $G(t) = e^{- \lambda t}$
subjected to the boundary condition $G(0) = 1$.

\paragraph{Connection to Poisson and Geometric distributions.}
There is a duality between discrete random variables and continuous distributions that I
suspect might be useful.

Let us consider the problem of finding how much time it takes until until
some pre-determined Poisson event happens.
For example, if we think about how many buses arrive at the bus station, this 
problem follows -- within a reasonable approximation -- a Poisson distribution.
The number of buses is large, but each specefic bus has a small probability
of arriving at the station at any particular exact instant.
This is precisely what the Poisson distribution models.

In any case, what we really wish to compute is
the expected time until the first bus arrives given that 
the number of buses that arrive at the station is distributed as Poisson
with some $\lambda$.
The bridge between continuous and discrete variables is 
given by the following statement:
\textit{the probability that the arrival time is greater than
some specific time $t$ is equal to the probability that no buses have
arrived at the station until $t$}:

\begin{equation}
P(T \geq t) = P ( \text{No buses until time } t ).
\end{equation}

\noindent
The rate in which buses arrive is given by $\lambda$, which means
that on an interval of size $t$ we have about $\lambda t$ buses on average.
This means

\begin{equation}
P ( \text{No buses until time } t ) = e^{- \lambda t} \frac{(\lambda t)^{0}}{0!} = e^{-\lambda t}.
\end{equation}

\noindent
Therefore, 

\begin{equation}
P(T \geq t) = e^{-\lambda t}
\end{equation}

\noindent
is distributed as an exponential of $\lambda$. 
The waiting time random variable has no memory if the distribution of buses is Poisson.

Because the sum of independent Poisson is also Poisson, the next time interval
$T_{2} - T_{1}$ where $T_{1}$ is the time until arrival of the first bus, and
$T_{2}$ is the time inteval until arrival of the second bus, is also
distributed as exponential with parameter $\lambda$.
In fact, every subsequent interval will be memoryless.
This is different than saying that the total time, $T_{1} + (T_{2} - T_{1}) + \ldots$, is
exponential. It is not; we will see the distribution later.

For the geometric distribution, the random variable $X$ measures the number of failed tosses
until the first success.
What is the waiting time until the first heads appears?
If we throw a coin with frequence $\omega$, then on a time interval of length $\Delta t$
we would have thrown $\omega \Delta t$ in total.
The distribution of $X$ is then

\begin{equation}
P(X + 1 \leq \omega \Delta t) = p  + p q^{1} + \ldots + p q^{\omega \Delta t - 1}.
\end{equation}

\noindent
Using the formula for the geometric series, we obtain

\begin{equation}
P(X + 1 \leq \omega \Delta t) = 1 - q^{\omega \Delta t}
\end{equation}

\noindent
which is an exponential distribution with mean $- \omega / \ln q$.
\textcolor{red}{TODO:} write about interpretation of this result.

\subsection{Expectation II.}
The expectation of continuous random variables is defined in 
terms of the PDF: 

\begin{equation}
\mathbf{E} [ X ] = \int_{\mathbf{R}} x \rho(x) \mathrm{d} x.
\end{equation}

\noindent
It is possible to arrive at this description by taking the continuous limit
of the discrete case definition.
We can see that it tells us something about the center of mass of the distribution.
It is also possible to define other measures of central tendency.

\paragraph{Median.}
For any random variable $X$, we say that $c$ is a median if $P(X \leq c) \geq 1/2$ and
$P(X \geq c ) \leq 1/2$.
Intuitively, the median is the value where half of the mass of the distribution falls
to left and the other half falls to the right.

\paragraph{Mode.}
For a discrete random variable $X$, the mode is defined as the maximum of the PMF, 
while for continous random variables the maximum of the PDF is called mode.
The mode represents the greatest mass density of all values of the distribution.

\paragraph{Uniform.}
Let $X \sim \text{Uniform}[a,b]$ then

\begin{equation}
\mathbf{E} [ X ] = 
\int_{a}^{b} \frac{ x \mathrm{d} x }{b - a} = 
\frac{1}{2} \left( \frac{b^{2} - a^{2}}{b - a} \right) = 
\frac{1}{2} ( a + b ).
\end{equation}

\noindent
The median is exactly equal to the mean and any value in the interval $[a,b]$ is 
a mode.

\paragraph{Gaussian.}
Let $Z \sim N(0,1)$ be standard normal.
From the definition we obtain

\begin{equation}
\mathbf{E} [ Z ] = \frac{1}{\sqrt{2\pi}} \int_{-\infty}^{\infty} z e^{-z^{2}/2} \mathrm{d}z = 0
\end{equation}

\noindent
because the integrand is an odd function and integration interval is symmetric.
The median and the mode are equal to the mean.

Given the transformation $X = \sigma Z + \mu$, it is true that $X \sim N(\mu, \sigma^{2})$.
In particular we obtain

\begin{equation}
\mathbf{E} [ X ] = \mathbf{E}[ \sigma Z + \mu ] = \mu
\end{equation}

\noindent
using linearity.

\paragraph{Exponential.}
The mean of the exponential distribution is

\begin{equation}
\mathbf{E} [ X ] = \int_{0}^{\infty} \lambda x e^{-\lambda x } \mathrm{d} x  = \frac{1}{\lambda}.
\end{equation}

\noindent
The exponential decay makes the mode equal to $0$; while the median 

\begin{equation}
1 - e^{-\lambda x} = \frac{1}{2} \, \implies \,
x = -\frac{1}{\lambda} \ln 2
\end{equation}

\vspace{.5cm}
As important as notions of central tendency, it is also possible to measure the spread 
of a random variable.

\subsubsection{Variance and Standard Deviation.}
We call variance the distance squared from the mean, it is defined as

\begin{equation}
\mbox{Var} [ X ] = \mathbf{E} \left[ ( X - \mathbf{E} [ X ] )^{2} \right].
\end{equation}

\noindent
Like the expected value, the variance is a single-number summary of 
the distribution of a random variable.
It tells us how spread the distribution is relative to its mean.

We can expand inside the brackets,

\begin{equation}
( X - \mathbf{E} [ X ] )^{2} = 
X^{2} - 2 X \mathbf{E} [ X ] + \mathbf{E} [X]^{2},
\end{equation}

\noindent
and apply the expectation, we obtain

\begin{equation}
\mathbf{E} [ ( X - \mathbf{E} [ X ] )^{2} ] = 
\mathbf{E} [ X^{2} ] - \mathbf{E} [ X ]^{2}.
\end{equation}

\begin{itemize}
\item $\mbox{Var} [ X + c ] = \mbox{Var} [ X ]$ for any constant $c$.
If we shift the distribution, it should not affect its variability.

\item $\mbox{Var} [ c X ] = c^{2} \mbox{Var} [ X ]$ for any real number $c$.

\item If $X$ and $Y$ are independent, then $ \mbox{Var} [ X + Y ] = \mbox{Var} [ X ] + \mbox{Var} [ Y ]$.
The proof needs the introduction of covariance, which will discuss later.

\item $\mbox{Var} [ X ] \leq 0$ with equality holding if and only if $P ( X = a) = 1$ for any real number $a$.
The proof of this property follows directly from the definition.
\end{itemize}

The standard deviation is defined as the square root of the variance and it is usually denoted by $\sigma$:

\begin{equation}
\sigma [ X ] = \sqrt{ \mbox{Var} [ X ]  }
\end{equation}

\noindent
The standard deviation is not linear, it satisfies:

\begin{equation}
\sigma [ \alpha X + \beta ] = | \alpha | \sigma [ X ] .
\end{equation}

\paragraph{Bernoulli.}
The variance of a Bernoulli random variable is

\begin{equation}
\mbox{Var} [ X ] =  \mathbf{E} [ X^{2} ]  - \mathbf{E}^{2} [ X ]
\end{equation}

\noindent
where the first term can be computed with LOTUS

\begin{equation}
\mathbf{E} [ X^{2} ] = \sum_{k = 0}^{1} k^{2} p( X = k) = p.
\end{equation}

\noindent
Together we obtain

\begin{equation}
\mbox{Var} [ X ] = p - p^{2} = p q .
\end{equation}

\paragraph{Binomial.}
A Binomial process is a sum of $n$ independent Bernoulli trials.
Since the variance of independent events is the sum of variances
we obtain

\begin{equation}
\mbox{Var} [ X ] = n p q
\end{equation}

\paragraph{Geometric.}
We need to compute

\begin{equation}
\mathbf{E} [ X^{2} ]  =
\sum_{k = 1}^{\infty} k^{2} p q^{k} = \frac{q (1 + q)}{p^{2}}
\end{equation}

\noindent
which implies

\begin{equation}
\mbox{Var} [ X ] = \mathbf{E} [ X^{2} ] - \mathbf{E}^{2} [ X ] =
\frac{q (1 + q)}{p^{2}} - \left( \frac{q}{p} \right)^{2} = \frac{q}{p^{2}}.
\end{equation}

\noindent
The variance is also the same if we wish to compute the distribution of
first success, because shifting by constants does not alter the result.

\paragraph{Negative Binomial.}
The negative binomial distribution is the sum of independent
geometric distributions. 
It readily follows that

\begin{equation}
\mbox{Var} [ X ] = r \frac{q}{p^{2}}
\end{equation}

\paragraph{Poisson.}
Using LOTUS

\begin{equation}
\mathbf{E}[X^{2}] = \sum_{k = 0}^{\infty} k^{2} P ( X = k ) = 
e^{-\lambda} \sum_{k = 0}^{\infty} k^{2} \frac{\lambda^{k}}{k!} = \lambda ( 1 + \lambda ).
\end{equation}

\noindent
This implies

\begin{equation}
\mbox{Var} [ X ] = \lambda ( 1 + \lambda) - \lambda^{2} = \lambda.
\end{equation}

\noindent
The Poisson distribution is the only distribution where the mean and the variance
have the same value.

In order to compute the variance of dependent random variables we need to introduce the 
concept of covariance.
Let us before that explore the interpretation of higher moments of a distribution.

\section{Moments and generating functions.}

\subsection{Moments.}

If we let $X$ be random variable with mean $\mu$ and variance $\sigma^{2}$, 
then we define

\begin{itemize}
\item The \textit{n}th moment of $X$ by \[ \mathbf{E}[ X^{n} ]; \]

\item The \textit{n}th central moment by \[ \mathbf{E} [ (X - \mu)^{n} ]; \]

\item The \textit{n}th standardized moment by \[ \mathbf{E} \left[ \left( \frac{X - \mu}{\sigma} \right)^{n}  \right].  \]
\end{itemize}

In general we say that $X$ is symmetric around its mean $\mu$ if $X - \mu$ has the same distribution
as $\mu - X$. In fact, we do not have much freedom, if $X$ is symmetric around a number $\mu$, then 
$\mu$ must be the mean since

\begin{equation}
\mathbf{E} [ X ] - \mu = \mathbf{E} [ X - \mu ] = \mathbf{E} [ \mu - X ] = \mu - \mathbf{E} [ X ]
\end{equation}

\noindent
which implies $\mathbf{E} [ X ] = \mu$.
Moreover, $\mu$ must also be the median since

\begin{equation}
P( X \leq \mu ) = P ( X \geq \mu)
\end{equation}

\noindent
follows directly from the definition of symmetry.

If $X$ has a PMF, then symmetry implies

\begin{equation}
F(x) = P ( X - \mu \leq x - \mu ) = 
P ( \mu - X \leq x - \mu ) = P( X \leq 2\mu - x) = 1 - F(2 \mu - x )
\end{equation}

\noindent
which after differentiating on both sides yields

\begin{equation}
\rho(x) = \rho( 2 \mu - x).
\end{equation}

\noindent
Functions with symmetry $f(x) = f(a - x)$ have maximum at $a/2$.
Therefore the above equation implies that the PDF has maximum at $\mu$,
which means $\mu$ is a mode of $X$.

For a symmetric distribution is possible to show that every odd
moment is zero. It follows simply from the consideration that

\begin{equation}
\mathbf{E} [( X - \mu )^{m} ] = - \mathbf{E} [ (\mu - X)^{m} ] = - \mathbf{E} [ (X - \mu)^{m} ]
\end{equation}

\noindent
for odd $m$.
This is enough evidence to study the odd moments as measures of skewness.
The first standardized moment is always zero, leaving us only the third moment.
Define then skewness as the third standardized moment, such that
positive skewness is indicative of having a long right tail relative to the left tail,
and negative skewness is the opposite.
Notice that exist asymmetric distributions whose odd central moments are zero.

\begin{center}
\textcolor{red}{TODO:Write about kurtosis}
\end{center}
\subsection{Moment generating functions.}
Moment generating functions are a useful computation device for 
calculating moments of a distribution.
Remember that the $n$-th moment of a distribution, denoted $m_{n}$, is defined
as

\begin{equation}
m_{n} = \mathbf{E} [ X^{n} ].
\end{equation}

\paragraph{Definition.}
Consider the polynomial $M(t)$ where the coefficient of the $n$-th power is
$m_{n}$:

\begin{equation}
M(t) = m_{0} + m_{1} t + m_{2} \frac{t^{2}}{2!} + m_{3} \frac{t^{3}}{3!} + \ldots. 
\end{equation} 

\noindent
By construction it follows that

\begin{equation}
m_{n} = \frac{\mathrm{d}^{n} M(t) }{\mathrm{d} t^{n} } \bigg\vert_{t = 0}
\end{equation}

\noindent
which means that we can generate the moments of the distribution if we know $M(t)$.
In principle, a random variable could have moments of arbitrary order, so this sum
is infinite. But if we use linearity of the expectation operator:

\begin{align}
M(t) &= 1 + \mathbf{E}[X t ]  + \mathbf{E}\left[ X^{2} \frac{t^{2}}{2!} \right]  + 
\mathbf{E} \left[ X^{3} \frac{t^{3}}{3!} \right] + \ldots
\nonumber \\
\nonumber \\
&= \mathbf{E} \left[ 1 + X t + X^{2} \frac{t^{2}}{2!} + X^{3} \frac{t^{3}}{3!} + \ldots \right]
\nonumber \\
\nonumber \\
& = \mathbf{E} \left[ e^{t X } \right]
\end{align}

\noindent
the sum colapses into an exponential.
Actually, linearity is not enough to guarantee that we can absord an infinite number of 
terms inside the expectation, but we apply it anyway without further justification.
If the sum converges, then $M(t)$ exists and we have defined the moment generating function.
The domain of $t$ is not of any importance in here, we just need $M(t)$ to be defined for
some interval on the real line.

There are two important facts about moment generating functions
that we are going to state without proof. 
There is, for each distribution, only one momentum generating function and vice-versa. And
given two \textit{independent} random variables, the momentum generating function of their 
sum is given by the product of the corresponding momentum generating functions -- a direct
application of LOTUS.

\subsubsection{Bernoulli distribution.}

We consider a Bernoulli random variable distributed according 
to $p$ and $q$.
The momentum generating function can be computed via LOTUS:

\begin{equation}
M(t) = \sum_{k = 0}^{1} e^{tk} \bigg[ kq + (1 - k)p \bigg] 
= e^{t} p + q.
\end{equation}

\subsubsection{Binomial distribution.}

Let $X$ be binomial, $X \sim \text{Binomial}[n,p]$.
From the definition, $X$ is the sum of independent Bernoulli trials.
It follows

\begin{equation}
M(t) = ( p e^{t} + q )^{n}.
\end{equation}

\noindent
In the last equality we used the binomial theorem.

\subsubsection{Geometric distribution.}

If instead we let $X \sim \text{Geometric}[p]$, the 
momentum generating function is 

\begin{equation}
M(t) = \sum_{k = 0}^{\infty} e^{tk} p q^{k} = \frac{p}{1 - e^{t} q}
\end{equation}

\noindent
Notice that the sum of two independent geometric distributions does not 
follow a geometric distribution. 

\subsubsection{Negative Binomial distribution.}
 
If we consider a sum of independent geometric distributions, we obtain
a negative binomial distribution, which implies

\begin{equation}
M(t) = \left[ \frac{p}{1 - e^{t} q} \right]^{n}.
\end{equation}


\subsubsection{Hypergeometric distribution.}

The closed form of the moment generating function for the hypergeometric distribution 
depends on hypergeometric functions, and are beyond the scope of this notes.
We will write an alternative version which does not have a closed form, but is
expressed as a function of $n$.

First consider

\begin{equation}
( 1 + y e^{t} )^{a} = \sum_{j = 0}^{a} {a \choose j} e^{jt} y^{j}
\end{equation}

\noindent
and 

\begin{equation}
( 1 + y )^{b} = \sum_{k = 0}^{b} {b \choose k} y^{k}
\end{equation}

\noindent
Then the trick is to notice that

\begin{align}
( 1 + e^{t} y )^{a} ( 1 + y )^{b} &= 
\bigg[ {a \choose 0} {b \choose 0}  e^{0 t} \bigg] y^{0} +
\bigg[ {a \choose 1} {b \choose 0}  e^{1 t} + {a \choose 0} {b \choose 1}  e^{0 t}  \bigg] y^{1} 
\nonumber \\
\nonumber \\
&+  \ldots + \bigg[ \sum_{k = 0}^{n} {a \choose k} {b \choose n - k}  e^{k t} \bigg] y^{n}  + \ldots 
+ \bigg[ \sum_{k = 0}^{a + b} {a \choose k} {b \choose n - k}  e^{k t} \bigg] y^{a+b} 
\end{align}

\noindent
gives the hypergeometric momentum genereting function if we normalize it appropriately:

\begin{equation}
( 1 + e^{t} y )^{a} ( 1 + y )^{b} = 
\sum_{n = 0}^{a + b} M(t) y^{n} { a + b \choose n }
\end{equation}

\noindent
such that

\begin{equation}
M(t) = 
\frac{(a + b - n)!}{(a + b)!}
\frac{\partial^{n}}{\partial y^{n} } 
\bigg( ( 1 + e^{t} y )^{a} ( 1 + y )^{b} \bigg)_{y = 0}
\end{equation}

\noindent
I this point I really do not know a useful application for $M(t)$. 
But it must be useful for poker.

\subsubsection{Poisson distribution.}

For $X \sim \text{Poisson}[\lambda, k]$ the momentum generating function is

\begin{equation}
M(t) = \sum_{k = 0}^{\infty} e^{tk} e^{-\lambda} \frac{\lambda^{k}}{k!}  = e^{-\lambda} e^{e^{t} \lambda}
\end{equation}

\noindent
Interestingly enough, notice that the sum of two independent Poisson's parametrized by $\lambda$ is
again Poisson.
To see why two dependent Poisson distributions would not be Poisson, consider the extreme case $Y + X = 2X$.

\subsubsection{Uniform distribution.}

Let us consider $U \sim \text{Uniform}[0,1]$. The general uniform parametrized by 
$[a,b]$ can be obtained by the variable redefinition:

\begin{equation}
(b - a) U + a = X.
\end{equation}

The moment generating function can be calculated with LOTUS

\begin{equation}
M(t) = \int_{0}^{1} e^{tx} \mathrm{d} x = \frac{1}{t} \left( e^{t} - 1 \right),
\end{equation}

\noindent
from which the moments $m_{n}$ can be directly derived:

\begin{align}
M(t) &= \frac{1}{t} \left( \sum_{n=0}^{\infty} \frac{t^{n}}{n!} - 1 \right) 
\nonumber \\
\nonumber \\
&= \sum_{n = 1}^{\infty} \frac{t^{n-1}}{n!}
\nonumber \\
\nonumber \\
&= \sum_{n=1}^{\infty} \frac{(n-1)!}{n!} \frac{t^{n-1}}{(n-1)!}
\end{align}

\noindent
and give

\begin{equation}
m_{n} = 
\frac{(n-1)!}{n!}
\end{equation}

\subsubsection{Exponential distribution.}

Let $X \sim \text{Exponential}[1]$. 
The general case with $\lambda$ can be obtained by the 
change of variables

\begin{equation} \label{expredef}
X = \lambda Y, 
\quad
\text{where }
X \sim \text{Exponential}[1].
\end{equation}

From the definition, we have then

\begin{equation}
M(t) = \int_{0}^{\infty} e^{tx} e^{-x} \mathrm{d}x = 
\int_{0}^{\infty} e^{-(1-t)x} \mathrm{d} x = \frac{1}{1 - t}
\end{equation}

\noindent
where $|t| < 1$ otherwise the integral diverges.
Notice that we have obtained a geometric series,
which means the moments can be read directly

\begin{equation}
M(t) = \sum_{n=0}^{\infty} t^{n} = 
\sum_{n=0}^{\infty} n! \frac{t^{n}}{n!}
\end{equation}

\noindent
so that

\begin{equation}
m_{n} = n!
\end{equation}

Using the redefinition \eqref{expredef} we obtain

\begin{equation}
m_{n} = \frac{n!}{\lambda^{n}}
\end{equation}

\subsubsection{Gaussian distribution.}

Let $Z \sim N(0,1)$ be a standard normal distribution.
The general case with parameters $\sigma$ and $\mu$ can be obtained 
from the change of variables

\begin{equation}
\frac{X - \mu}{\sigma} = Z.
\end{equation}

The moment generating function gives

\begin{align}
M(t) &= \frac{1}{ \sqrt{2 \pi } } \int_{-\infty}^{+\infty}
e^{t x} e^{-\frac{x^{2}}{2}} \mathrm{d} x
\nonumber \\
\nonumber \\
&= \frac{1}{ \sqrt{2 \pi } } \int_{-\infty}^{+\infty}
e^{- \frac{(x - t)^{2}}{2}} e^{-\frac{t^{2}}{2}} \mathrm{d} x 
\nonumber \\
\nonumber \\
&= e^{- \frac{t^{2}}{2} }.
\end{align}

\noindent
In the first equallity we have completed squares, and in the last equality
we used the fact that a standard normal integrated over its domain gives $1$.

The same technique from previous sections can be applied -- just expand $e^{-t^{2}}$ in
Taylor series and collect the coefficients. We obtain

\begin{equation}
m_{2n} = \frac{(2n)!}{2^{n} n !}
\end{equation}

\noindent
while every odd moment is zero.

\subsection{Probability generating functions.}

Let us consider the redefinition $t = \log u$ inside $M(t)$:

\begin{equation}
M( \log u ) = \mathbf{E} \left[ e^{ X \log u } \right] 
= \mathbf{E} \left[ u^{X} \right].
\end{equation}

\noindent
Using LOTUS this expression is equivalent to the following polynomial

\begin{equation}
\mathbf{E} \left[ u^{X} \right] = 
\sum_{k} u^{k} p ( X = k) = 
p_{0} + p_{1} u + p_{2} u^{2} + \ldots
\end{equation}

\noindent
where we have abbreviated $p_{0} = p(X = 0)$, $p_{1} = p(X = 1)$, $p_{2} = p(X = 2)$, \textit{etc}.
This polynomial is called probability generating function and we are going 
to denote it by $H(t)$:

\begin{equation}
H(t) = \mathbf{E} [ t^{X} ].
\end{equation}

\noindent
Because it is related to $M(t)$ by a redefinition, and it is a way to compute the probabilities that we have
used to define our random variables, we are not going to rederive various expressions for it; just observe that the 
probability generating function for independent random variables is the product of probability generating functions -- 
a consequence of LOTUS.

The usefulness of $H(t)$ is best illustrated in practical computations.
Until now if we wished to compute the probabilities of obtaning a $10$ when throwing six dices, 
we would have to count all configurations which sum to give $10$,
a horrible task.
Moreover, even if we succeeded in such counting, we do not 
obtain information about configurations with $11$, $12$, or $4$ for that matter.

\subsubsection{Dice polynomial -- Dodo.}
Let us consider a fair dice. From the definition we can write

\begin{equation}
H(t) = \frac{1}{6} \left( t + t^{2} + \ldots + t^{6} \right).
\end{equation}

\noindent
And using the partial sums of the geometric series:

\begin{equation}
H(t) = 
\frac{t}{6} \left( \frac{1 - t^{6}}{1 - t} \right).
\end{equation}

\noindent
For $N$ independent dices, the polynomial becomes

\begin{equation}
H(t) = 
\left( \frac{t}{6} \right)^{N} \left( \frac{1 - t^{6}}{1 - t} \right)^{N}
\end{equation}

\noindent
and in principle we have computed any relevant probability of the problem. Just
derive $H(t)$ a number of times and set $t$ to zero.
The advantage is that this problem can now be very easily programed.
In particular, the Taylor series of $H(t)$ for $N = 5$:

\begin{align}
H(t) &=
\frac{t^{5}}{7776} + \frac{5 t^{6}}{7776} + \frac{5 t^{7}}{2592} + \frac{35 t^{8}}{7776} 
+ \frac{35 t^{9}}{3888} + \frac{7 t^{10}}{432} + \frac{205 t^{11}}{7776} + \frac{305 t^{12}}{7776} 
\nonumber \\
\nonumber \\
&+ \frac{35 t^{13}}{648} + \frac{5 t^{14}}{72} + \frac{217 t^{15}}{2592} + \frac{245 t^{16}}{2592} 
+ \frac{65 t^{17}}{648} + \frac{65 t^{18}}{648} + \frac{245 t^{19}}{2592} + \frac{217 t^{20}}{2592} 
\nonumber \\
\nonumber \\
&+ \frac{5 t^{21}}{72} + \frac{35 t^{22}}{648} + \frac{305 t^{23}}{7776} + \frac{205 t^{24}}{7776} 
+ \frac{7 t^{25}}{432} + \frac{35 t^{26}}{3888} + \frac{35 t^{27}}{7776} + \frac{5 t^{28}}{2592} 
\nonumber \\
\nonumber \\
&+ \frac{5 t^{29}}{7776} + \frac{t^{30}}{7776} 
\end{align}

\noindent
was obtained with the Python code below:

\begin{verbatim}
	import sympy as sp
	
	# Number of dices
	N = 5
	
	t = sp.symbols('t')
	A = (1 - t**6)
	B = (1 - t)
	H = ( t / 6 ) * (A/B)
	H = H ** N
	
	sp.pprint(H.series(t, 31), use_unicode = False).
\end{verbatim}

\noindent
As a sanity check, the probability must start from $5$, since we have five dices.
It must also end at $30$, the case where all five dices give $6$.
The probability of $30$ is $1/6^{5} = 1/7776$ in agreement with the coefficient of
$t^{30}$. Other cases are also easily countable, like the probability of obtaning a sum of $6$ in
$5$ dices, which comes from a configuration $21111$ with $5$ possible places for the number $2$.
The probability of this configuration is $5 / 6^{5}$ again matching the coefficient of $t^{6}$.

The advantage of this polynomial becomes apparent if we wish to compute the probability of obtaining 
a sum of $13$ for example. Counting such configurations is a complicated task
(\textcolor{red}{TODO}), but the polynomial gives readily that there are $420$ configurations, 
and a probability of obtaining such configuration equal to $35/648$.

As a final observation, notice that the sum of all coefficients gives $1$.
This must always be the case since

\begin{equation}
H(t = 1) = \sum_{k} p_{k} = 1
\end{equation}

\section{Joint Distributions.}

\end{document}
